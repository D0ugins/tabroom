\documentclass[12pt]{report} \usepackage {fullpage} \usepackage{times}
\renewcommand{\arraystretch}{1.5}

\author{Chris Palmer}

\begin{document} \normalsize

\title{Managing Tournaments with Tabroom.com} \maketitle

\date

\tableofcontents

\newpage

\chapter{Welcome to Tabroom}

Welcome to the tabroom.com system.  This system allows you to run and panel
a speech tournament from the beginning of self-registration on the web to
the end of the results and publishing them on the web.   Hopefully you will
find it useful; I've written it entirely so I had something I myself could
use, not for profit or reward, of which there is precious little in the
world of forensics anyway.

\medskip

The reason I wrote my own program is because existing software was either
expensive, or did not implement our local rules correctly, or did not allow
me to use more than one computer at a time for data entry, or some
combination of the three.   So this software does my rules, allows
unlimited numbers of computers (since it just runs over the web) and it's
free since I wrote it. 

\medskip

The format of this manual is similar to the format of the software.  When
you log into the tournament director's side of things, you will have a row
of drop down menus across the top of the window.    As your tournament
progresses, you will find yourself marching from left to right and top to
bottom in each of these menus; this is not by accident.   If you're ever
unsure of where to go next, the best rule of thumb is to use the next
option over from the one you were just using; it may just have what you
were looking for.


\chapter{League Admins: Creating Tournaments}

To create a new tournament, you must be a league administrator; if you are
not a league administrator for your league and should be, please contact
either someone who is, or Chris.

\medskip

To create a new tournament, do the following:

1.  Enter the system but do not go into a tournament.  Instead, under
League in the menu items, pick Tournaments

2.  Click the Add New button on the bottom.

3.  Fill out the tournament dates and registration dates; for a one day
tournament just make the start date the same as the end date.

4.  Put the contact email for the tournament director in the box; this
person must already have a tabroom.com account.

5.  (Optional) Using the Clone Tournament box, you can import settings
and events from a previous tournament; use this if your current tournament
is close to or identical to a previous tournament's rules and setup; it
will save you time. 

Click save, and the tournament is created.  If you imported tournament
settings, you may want to look into the tournament itself, under
Setup::Events and such, to make sure everything copied the right way.

\section{Locations \& Sites}

Instead of having to re-enter all your rooms year to year, the system will
track them for you.  Therefore you must add sites to your tournament before
it will allow you to enter rooms into a tournament.  Create sites under
League::Locations and add the names and directions to your sites -- the
directions will appear for people registering for the tournaments.  Then
when you create a tournament, or afterwards, you can add site(s) to the
tournament.

Support for multiple sites (keeping judges to one site, etc) is not yet
functional.


\chapter{Tournament Directors: Setup}

First, you'll need to get access to your tournament.   Log into the site
that corresponds to your league.  If you are set up with access to a
tournament, you will have a "Tournaments" option under the Tourneys menu
item.  Select it, and you will see all your tournaments listed.  Select the
tournament you wish to panel. 

Tournament setup is where you establish your event policies, your
tournament tabulation methods, and your schedule.   You must do this before
your tournament opens online registration, as event policies have a direct
bearing on how your schools will register for your tournament.   If your
league's public website is managed by tabroom.com, it is a good idea to do
this at the start of the season, as the public website will list out your
tournament's events and other information, and it's a good idea to have
that available as soon as you can. 

All of these functions are on the Tournament side of the software, under
Setup.  Going from the top to the bottom of the setup menu is generally the
easiest method.  
 	
\section{Tab Settings}

The first menu item under Setup is Settings.  This allows you to set a wide
variety of tournament wide options for registration, tabulation and the
like.   

\begin{itemize}

\item Speech panel sizes:   Sets the minimum, default and maximum panel
size allowed when paneling preliminary rounds for Speech.  Elimination
round sizes are determined at the time you do breaks.

\item Congress chamber sizes:  Same as speech panel size, just for Congress
events.

\item Track housing requests:  This enables the subsystem which allows the
registrants to also request housing from tournaments which offer it.  More
settings and details can be found under Setup->Housing when this is
enabled.

\item Ask for Qualifying Tournament:  This box, primarily used for NYSFL
and NJFL State Tournaments, will provide a box for the registering coach to
enter which tournament the student used to qualify for their entry.

\item Double Entry Allowed: This will determine what sort of double entry
you permit in your tournament.  None means that no student will be allowed
to enter in more than one event.    Two events allows a maximum of two
events per student.   By flight means that you will enter your events into
two flights, and each student will be allowed only one event per flight
(unless the event category allows for more than one: see Creating Events).
And finally, unlimited allows unlimited double entry.

Please set this to the \emph{most liberal} setting for your tournament;
individual events and event classes will allow you to restrict double entry
further for certain events and event classes.   This setting only serves as
a baseline.

\item Hide Speaker Codes from Registrants: 	 Does not permit coaches to see
the speaker codes when they register; otherwise they will be able to see
competitor codes during registration.

\item Elimination round paneling method:  This setting determines how
students will be assigned to elimination rounds.  The first option snakes
according to the ranking of the student out of their last elim rounds or
prelim rounds, and does not adjust the bracket for school conflicts.  The
second option will adjust for students in the same schools ONLY within the
same criterion you specify in the second selection box; you can swap kids
with the same cumulative score out of prelims, the same seed order out of
the last elimination round, or the same overall cumulative score so far.
The third option will break the bracket to preserve students competing
against the same schools no matter what, even if no students with the same
cumulative scores or seeding are available.  The last option seeds students
into elimination rounds following the bylaws of the National Catholic
Forensic League.

\item Judges may see the same event twice:  Allows judges to judge the same
event more than once.

\item Use the Judge Qualification System: 	Uses the judge rating system to
permit registering coaches to rate their judges according to a scale that
you set.  See Judge Qualifications below.

\item Allow judges from the same school on a panel:  Allows judges from the
same school to sit on the same judge panel when more than one judge judges
a round.

\item Judges may judge their own students: 	 Allows judges to judge their
own students.  This is also a per event setting for events which permit
judges to judge students for their own school; if you want this to be true
only in some events, use the setting under the event, not this setting.

\item Paneling weight for avoiding same schools \& Paneling weight for
avoiding same students:   This helps set the formula for how you want to
panel the tournament.  Sometimes paneling so that students do not compete
against their schoolmates is more important; sometimes paneling so that
students do not compete against other students more than once is more
important.  Set these numbers higher if they are more important, lower if
they are less important.   For instance, if you wanted school hits to never
happen unless there were no other choice regardless of the number of times
students competed against the same student twice, set same-schools to 10
and same-students to 1.
	
\item Fine for drops after registration closes:  This is a fine that is
automatically assessed every time you drop a student after the close of
registration.  Set to 0 for none.

\item Fine for adds after registration closes:  Same as above, but for
manual additions to the tournament.

\item Fine per round for no-show judges:   When you replace a judge during
the tournament, there are two options: one simply removes the judges, and
the other removes them and also applies a fine to that judges' school.  Set
the amount of that fine here.

\item Track housing requests: 	This option will have the tournament
software track requests for housing as well as registrations.  See Housing,
below.

\item Drop best rank:  This option will not count the lowest rank the
student receives when tabulating preliminary rounds into elimination
rounds.  

\item Drop worst rank:  Same as above, but leaving out the highest rank.

\item Truncate all ranks to size of the smallest panel:  This option will
calculate the smallest panel in the tournament, and raise any rank that is
higher than this number to this number.  So, if there are panels of 5 and 6
students, all students getting a 6 will be counted as a 5 so they are not
penalized for landing in rounds 

\item Truncate all ranks to:  Truncates all ranks to the specified number.
This can work in conjunction with the above setting; the rank will be
truncated to the lowest possible rank, so if this is set to 5 and the above
option is set, and there is a panel of 4, 

\item MFL Time Violations (auto-drops one rank):   This will automatically
add one rank to each rank if a round is marked as overtime, in accordance
with MFL rules.

\item Use MFL flex finals (6 or 7 if ties): 	This will advance 7
students to final rounds if there is a tie on rank between 6 and 7, only
breaking the tie if it extends to 8th place or beyond, in accordance with
MFL rules.

\item Honorable Mentions (ties out of prelims): The awards results sheet
will list out students who tied on ranks with students advancing to the
final round who themselves did not advance, for recognition in the final
round.

\item No-shows forbidden from breaking: 	Students who are marked as
no-shows in a round cannot break to finals no matter what their cumulative
score is if this option is checked.

\item Sweepstakes Method:    Currently there is only one method;  the
NCFL/MFL  6-rank method.  This is used for any method based on ranks.

\item Sweep ranks in prelim rounds: 	Counts all ranks given in
preliminary rounds.

\item Sweep based on event classes:   Sweep event limitations based on
classes, not individual events;  so instead of counting the top 2 in each
event, it counts the top 2 in each event class.

\item Sweep ranks in elim rounds: 	Counts all ranks given in elimination
rounds.

\item Sweep top \# of entries per event/class: 	Limit on how many students
per school per event or class are counted.

\item Sweep ranks in final round: 	 Sweeps ranks given in final rounds.

\item Sweep top event/classes:  Limit on how many events or event classes
are counted per school.

\item Sweep elim round placement: 	  Sweeps based on the place within an
elim round, not the ranks given.

\item Sweep additional wildcards:   Add this many of the next top scorers
after the rest of the limit rules have been applied.
		
\item Sweep on final overall ranking:  Gives sweep points based on the
final placement of the student.

\item Points per finalist:   Gives this number of sweep points per final
round appearance.

\item Points per elim round: Gives this number of sweep points per elim
round appearance.

\item Bid percentage:   Give a bid to students in this top percentage of
each league sponsored event.
		
\item Give bids to students tied on ranks: 	Give bids to students tied on
ranks with any other student who receives a bid.

\item Bid maximum cume: 	Give bids to any students at or below this
cumulative score.

\item Bid if you break to: 	Give bids to students who advance to this
round.

\item Notes that appear below judge registrations:  This text will appear
when the coaches register judges; put any warnings and notes about judge
restrictions and limitations here.

\item Disclaimer that appears before registration:  This text will appear
and have to be agreed to by any coach who registers to the tournament.  Put
any disclaimers or important notices you want to make sure that coaches
have agreed to here. 

\item Message appearing on the invoices:   This message appears on all
printed receipts and invoices for the tournament; it can include payment
information or the like.

\item Message appearing on housing requests: This message appears on the
screen used to register selections for housing requests, and can include
any rules or warnings for housing requests you may wish to include.

\end{itemize}

 	
\section{Schedule}

Then you must create your tournament schedule.  There are two concepts here
-- rounds, and time slots.   Time slots are the blocks of time you portion
off for various aspects of your tournament.  Rounds are where you schedule
rounds into those time slots for a given event. 

A round can only be in one time slot.  So if you have a round of one event
that goes from 10 AM to 11 AM, and another event that goes from 10 AM to 1
PM, you must create two different time slots, even though they overlap, so
that you can schedule rounds for each event into them. 

Please note when creating time slots that the computer uses these to assign
judges and rooms to panels.   However, it will not recognize overlaps.  If
you assign Round 2 to run from 11:00 AM to 12:00 PM, and then assign round
3 to run from 12:00 PM to 1:00 PM, the computer will not permit the same
rooms or judges to be used in both rounds, which is usually not the
intended effect.  Instead, have round 1 run from 11:00 AM until 11:55 AM,
so the time slots do not overlap.  

 	
\subsection{Creating Time slots}

First, create time slots for every block of time in your tournament,
assigning a name and a starting date and time and an ending time for each
time slot.  Time slots may not span multiple days, because that's just
insane.  I refuse to be a party to any tournament that intentionally
schedules a round that extends past midnight.  I'm sorry, I just won't do
it.

 	
\subsection{Assigning Events to Time slots}

If you are running a simple tournament without flights or a complex
schedule, you may want to only add in time slots for your speech
preliminary rounds first, and then hit the "Auto Schedule Prelims (Speech
Events Only)" link at the top of the page;  this will save you from having
to put speech preliminary rounds into each time slot.   You must have
created the events first (see next section.)

Otherwise, you can hit the "rounds" button next to each created round and
schedule which events have rounds in that time slot using that slot.   Each
round will be numbered in order based on when they are created, so be sure
to schedule events into Round 1 first, and then Round 2 and so on, so your
tournament isn't tragically confused since Round 3 comes before Round 1. 


 	
\section{Creating Events}

All events must belong to an \emph{event class}, and all event classes must
belong to an \emph{event group}.  Groups are events that can share judges
in common between them.  Events in the same groups have the same judging
obligations on the parts of the schools, and the same judging fees. 

Event classes are events that share the same double-entry, judge quality,
and sweepstakes policies.   If you need to change any of these things
between events, you must create a separate class or group for the events
that differ.

 	
\subsection{Event Groups}

First, create event groups on the Event screen at the bottom for each event
grouping you wish to have.  Typical event groups will be LD, Policy, IE and
Congress, for example.   Set the fees involved in judges and judge
obligations for the event group; the fee for a hired judge is for schools
that explicitly hire judges, while the short judges is a fee for either
judge groups that do not have hired judges, or schools failing to either
explicitly mark that they want to hire judges, or who are short when the
hired pool is all taken up.

The fees for judges can be based either on an uncovered entry, which means
that every student who does not have a judge will be charged a set amount,
or a fee for a missing judge, which means that no matter if only 1 student
is uncovered or if a full judge obligation is uncovered, the school will be
charged this fine for a full judge. 

 	
\subsection{Event Classes} Then, for each event group, create one or more
classes depending on how you wish to set up your tournament.  If, for
example, you are using the judge quality system, you may want to divide IE
between Address events (Extemp, Oratory) and Interp events (Duo, DI, HI) so
that judges can be rated separately in these events.   You also will want
events that have double entry restrictions (Congress, Limited prep events)
to be in their own class.

 	
\subsection{Events}

Create each event by clicking on the Create New Event button.  Give the
event a name and a 2-4 letter abbreviation.   The Event Type will signal
how the event is to be paneled:  Speech and Congress events are paneled
according to the settings for room size; Debate events are currently
ignored or imported into TRPC.

The Event Cap will limit the amount of entries and registrations accepted
by an event.     The School cap will determine a limit on each school's
entry in a given event.  Both caps can be set at the same time.  The
Waitlist checkbox will, if checked, allow coaches to add students to a
waitlist whose order is tracked if an event is full, or if they have
reached their school's cap in an event. 

Entry determines how people enter for the event; by themselves (most
events), in pairs (Duo and Policy Debate) or in teams of more than one,
which are given names and do not have students assigned to them explicitly.
Event Class shows which event class the event is a member of.   Speaker
Codes lists the lowest speaker code that the event has; each speaker will
get the next available code.   Be sure to leave enough room between
starting speaker codes, lest they run into one another; however, no two
students at a tournament will receive the same code even if two events run
into each others' ranges.

The entry fee is the per-entry fee.   Judges can judge their own school
will permit judges to see their own students in this event.   And finally,
League sanctioned event designates which events should be used in
calculating bids; bids will only be given in events where this is checked
off.

The ballot upload allows you to upload a sample ballot file.  And the event
policies will be shown when coaches register students in this event, so you
can use this space for any emphasis or last minute changes you wish to
make.

Below the event information, the event schedule is listed;  each round and
the time slot, type, and label of the round.  For more on rounds, see
Schedule above.

 	
\section{Rooms}

Rooms are divided into Sites;  your league should already have imported the
Sites that your tournament will use into the tournament.  Be sure that
different sites are actually different; judges will not be assigned to
judge events in two different sites, assuming that there is no access
between them.   If you want to divide your judging pool geographically,
then using two different sites is the way to go.

Enter the rooms one at a time; be sure to make the names descriptive, as
these names will appear on the schematics and postings for the event.
The lines about room quality are currently not used.

If you have a room that is only useful for part of the day, block it off
from time slots under Blocks.

 	
\section{Money}

Money is a convenient screen to set all of your tournament charges for the
downloadable invoices for the coaches.   You can use this screen to set all
your competitor fees in the same place, as well as judge fees and
penalties, and penalties for late drops and adds.   At the bottom of the
screen you can also set a standard fine with a time bound.  Say you are
charging a school fee -- set the fee here, and make the time bound the
entire length of registration.   You can also charge a fee for schools that
register past a certain date; set the date and time it should begin here,
and the amount, and it will be applied to any school that registers.

 	
\section{Access}

Access will define which user accounts have access to this tournament.  It
also allows you to set the dates of the tournament, and the dates when
online registration is available.   You can also use this screen to upload
an invitation and a bill packet to the public tournament information
screen, if your league also hosts its public site on tabroom.com.

 	
\section{Housing}

The Housing screen allows you to handle housing requests for students and
judges.  It is not yet complete pending fixes that were inspired by this
year's Lexington Winter Invitational. 

 	
\section{Tiebreakers}

Tiebreakers allows you to set the order and weight of the methods used to
advance students from round to round.   You can set the type of tiebreaker,
the priority, the scope, and the multiplier.

The way it works is you translate the rules in order to the terms the
computer uses.   Usually, for advancing students out of preliminary rounds,
the first method used is the cumulative score of the student's ranks from
prelims.   So you would give Cumulative a priority of 1, the first
tiebreaker, and the basis would be Prelims.

To advance from an elim in sudden death, you would use the Cumulative Score
or whatever other tiebreakers you desired, and set the scope from the Last
Elim.

The multiplier is in case you want to combine scores for a tiebreaker.  For
example, if a tournament wanted their final overall ranking to be based
first on final ranks, and then also on elimination rounds ranks, but wanted
to weight the final to count twice as much as the other eliminations, you
would set two tiebreakers.   First, select Cumulative, with priority 1,
basis Finals, and multiplier of 2.    Then select Cumulative, priority also
of 1 (since you want the two added together), bases of Other elims, and
multiplier of 1.  Then the first tiebreaker will calculate that score for
you.

Be sure that when you set two tiebreakers to be of equal prority that
they'll make sense added together; it would make no sense to add cumulative
scores and reciprocals and quality points; generally you'll only want to
add together scores of the same type.

Judges Preference is only usable in the final round.    Quality of
Competition calculates the average scores of the competitors each person in
a tie competed against; this tiebreaker generally only is useful in events
of 40 students or more, since in smaller events each student competes
against a very high percentage of the field, reducing the usefulness of
this tiebreaker.

 	
\section{Flights}

Flights are used to determine double entry.     You can create flights
specifically for the case that students are allowed to choose to enter one
event each in two or more lists.   Each list becomes a flight, and students
will not ordinarily be allowed to enter in two events of the same flight.

However, if you wish to allow entry in two events of the same flight, you
can set this permission in each event's settings screen described above.

Flight X is for special events which do not permit double entry at all;
events such as Congress or Policy Debate which have no off time and
therefore do not permit any double entry go into Flight X.  You can also
forbid events of a certain event class from having no double entry if you
wish. 

 	
\section{Judge Quality}

Judge quality is a mechanism that allows the coaches to rate their own
judges in terms of experience and quality.  You can set up whatever judge
rating system you choose.  Generally speaking the labels are letters
starting with A.  However, it is important to type in a description of what
you mean by A judge in the box next to the field, so that way coaches
registering are on the same page as you are when registering.

The Use Me box indicates whether or not judges of that class are actually
going to be used.   You can set up a judge class for judges who do not
judge a certain category of event, and check this box so that those judges
are not paneled at all.

\chapter{Registration}

The Registration menu allows you to handle tournament functions up to and
including when schools arrive and register for your tournament.   This is
where you can see event totals, find out financial information about your
tournament, make changes after the registration deadline on behalf of
coaches, and confirm that schools have paid and have registered.


\section{Information}

The first menu under Registration allows you to view, or print, a variety
of interesting information relating to your tournament.     Print
Registration Packets and Print Invoices allows you to print a copy of
everyone's complete registration and invoices, which is useful in preparing
for a tournament.     The Registration Coversheet is a listing of all the
schools in your tournament, along with their total registration fees, so
that you can check them in one by one as they register.     Print Schools
and Codes is useful if you have assigned letter codes or number codes to
your schools and use them for a purpose within the tournament (see League
for how to set school codes).  

Hired Judge Numbers will tell you how short or over you are your overall
judging obligations.    The Double Entry Report will tally up how many
students are double entered between which events.  And finally, the
Financial Statement will let you know the sum total of money you are owed,
and from which sources.

\section{Registration}

This screen allows you to manage the process of registering people as they
appear at the tournament site and pay you.   The checkbox for Registered is
a simple way of tracking who has appeared and who has yet to appear.   Then
you can type in the amount they have paid you, so you can track who owes
you money still and who does not.   Then click on Save next to the school
if you are finished.   If the school has any changes, then click on Entry
to access their school entry information screen.   You can also print them
a new invoice by clicking on Invoice and printing the PDF file that
appears.  
 	
\section{Schools}

Or, Everything you wanted to know about your entry but were afraid to ask.
This is the section wherein you will change, alter and clean up after
schools that have entered in your tournament.   Select a school from the
pull down menu and you will see a dizzying array of options.   You can
change the name of the school, if you want to make school names shorter or
somesuch, which can be nice if a name goes on forever, since printouts are
affected by this.   

Under the school name you can edit the entry of a school program.  This is
where you will want to make drops, adds and changes for a school as they
request them.  	During Registration, people can change their own entries
via the website.  However, after the close of registration, any drops or
changes need to be affected by the tournament staff.     To make changes to
the entry of a school, click on View/Change Entry.

A full listing of students and judges will appear.  To drop a student,
select Drop next to their name.   If you drop a student after the
tournament has been panelled, be sure you check the schematics for that
event afterwards, and fix any unbalanced panels. 

To change their name, click on Edit.   Do not simply change their listed
name; if you want to switch to another student, select the name of the
student from the pulldown menu to the right next to Code Switch.  That way
State Bids and the like are tracked appropriately.   

If the student does not yet exist in the system, select their event on the
menu at the top of the View/Change Entry screen for the school.  A link on
the right will allow you to add a student to the school, which you can then
add to the tournament or code switch another competitor to. 

 	
\subsection{Late Adds}

If you add a student after you have panelled the tournament, you will have
to manually add them to the tournament panels.     After you add them to
the tournament, a screen with their competitor information should appear,
with links to Assign Manually to Round next to each round.  Click on each
round in turn to select a panel to add the student to.  This screen with
adds will not allow you to assign students to events where they may not be
judged by the judge, and it will order the panels available by best fit
based on how many students from the same school or how many students the
new add has competed against in other rounds.   Pick the topmost round that
makes sense, then click on the student's entry to add them to other rounds. 


\section{Events}

The Events menu allows you to see the entry numbers by event, and a full
listing of how many students are in each event.  If you click on the name
of the event, you can see a listing of the students entered;  click on
Show/Hide Schools at the upper right to list them by school.   If you click
on the name of a student, then you can edit the details of that competitor
within the tournament as well. 

The totals on the main event screen do not include drops, but dropped
competitors will be listed and marked as such when you go into the event to
view details. 

\section{Entries}

Inevitably, about a million times a day you will have to look up a
competitor code to find out what school the kid is from, what their name
is, or whatever.   This is where to do it; simply type in the code and you
will tell who the competitor is, where they are from, and what their panel
assignments are.   You can click on the ranks they have gotten in order to
go to the rank entry screen (See Tabbing).   You can also click on Move to
move the kid to another panel, or drop the kid here.   Click on the
student's name to see if they are entered in other events. 

\section{Drops}

The Drops menu simply makes a listing of all the students you've dropped
from the tournament.   This might be useful for clearing up confusion on
the part of judges who were waiting for a student to appear, if you want to
post this list somewhere at your tournament. 

\section{Judges}

This screen is to judges what the Events screen is to competitors.  You can
see here what judges are entered in the tournament and in what numbers.  If
you view the chart, you can see the judges, how they are rated (if you are
using the Judge Qualification System), and any special requests that the
registering coach may have made about those judges.    You can also click
on the judges individually and find out interesting things about them.  

\subsection{Special Purpose People} If you have a judge who is special
purpose; in other words, performing a job that is not simply judging
rounds, you will want to do two things.  First, mark the judge as Inactive;
this designates a judge not used in the judging pool.  Secondly, you can
then type in a note under Special:.  This note will be listed in addition
to any judging assignment a judge has on their school's registration sheet.    

The Notes: field is notes that the registering coach wrote to you, and will
not appear on registration materials. 

\subsection{Judge Constraints}

The most useful thing here is strikes for judges.   You can strike a
judge against a school or student other than their own, an event, or a
block of time.   Click on Constraint on the right of the judge screen to
add a strike.

\subsubsection{Pre-blocking judges for elims}
 
Sometimes if you a running a larger tournament with many elimination rounds
it can be good to block judges who are clean in certain events from judging
that event during preliminary rounds; this way you can be certain they can
be used in elimination rounds, when good judges are at a premium.    If you
go to Panel:Judges, on the bottom left hand of the screen you will be able
to select judges event by event and block them from judging prelims only of
certain events; the screen will only show you judges who are clean to judge
these events.   
  

\chapter{Paneling}

 	
\section{Schedule}

The first step to paneling your tournament is defining the schedule of the
rounds; be sure this is done, or you won't get very far in paneling your
tournament.  See "Schedule" under Setup above.

\section{Assigning Panels}

After you have assigned rounds to the time slots, you're ready to panel the
students.   Go to Panel:Assign.    Next to each event, there will be a
listing of how many rounds are scheduled, and then a button to Panel.
Click on Panel for the event you wish to panel.  The next screen will give
you options for panel sizes.  Select the one based on how large you want
your panels to be and how many rooms you have.   These options will be
determined based on the minimum, default and maximum you specified in
Settings.

Sometimes, an option that you may want will not be there; sometimes if
there are two possible combinations involving rooms of 6s and 7s, no option
will appear; or you want to break your own rules about minimums and
maximums in just one case.  If this is so, just type in the number of
rounds in the Number of Rounds box, and that will override any selections
about panel size you have made in the check buttons.

Then let it panel, and it will present you with a schematic of the paneling
it has done.  Repeat for each event. 

\section{Setting Speaker Order}

The system has a currently rudimentary method of setting speaker order.
It will assign a random speaker order to the contestants and then print the
contestants in speaker order.   However, the system does not at all try to
make the speaker order "fair"; it is completely random.   It will, however,
deal with double entered students somewhat rationally, placing them towards
the beginning of one round and the end of another.    This system works
better if you have small numbers of double entered contestants in each
round; if you have more than 3, it tends not to be as useful.

To set order under these limitations, go to Panel:Order, and click on Set
Speaker Order.
 	
\section{Paneling Judges}

To panel judges, the first step is to remove anyone from the judging pool
whom you may be using for other purposes, such as the Tab room or Prep room
or whatnot.   Do this by going to the full judge listing under
Register:Judges.    Click on a judge you want to assign to another task,
and edit their record.  Put the assigned task under "Special Notes":  this
will appear on their school's registration sheet, so they are not wandering
around aimless and lost through the hallways of your tournament.  Then,
uncheck the box that marks a judge as "Active".  If they are not marked as
active the system will not panel them to judge rounds.  

Then, the next step is to select judges for special events, such as
Congress, if you wish to do this manually.   Go to the Panel:Schemats
screen, and look at the schematics for Congress.   Click on the panel
letter and you will see a large list of the students in that chamber
together with a long list of judges at the bottom.  Select a
Parliamentarian and your Congress judges out of that list, and assign them
to the panel manually.   

Then just let the system assign the rest of the judges.  Go to
Panel:Judges, and on the right select Speech to panel, and click Select.
Have the system panel all prelims with 1 judge per panel, but do not have
it clear old assignments -- that will remove your hand selected judges.
Then click Panel. 

Then you'll want to relax a bit and go have a cup of coffee, this part
takes a while but it will finish.   Once it does finish it will let you
know if you had any panels without a free judge.   Usually this does not
happen if your tournament numbers are OK.

The judge paneler is sometimes overambitious at using judges who are
difficult to assign to panels.  It works by taking the most struck
panel (the one with kids from the largest number of entries in that event)
and then finding the most struck judge who is clean to judge that
round.  As a result, judges from large schools have a tendency to be
overused.  

For example, it may give some judges 3-4 rounds and some only 1 or none.
You can rebalance the judge burdens by going to Panel:Judges, and clicking
on Show Judge Chart on the upper left.  There will be a button to Rebalance
Assignments in the upper right of this screen.  

This screen will also show you which judges are judging which events and
when, if you want to get an idea of how unbalanced things might be.

 \section{Viewing and Manipulating Schematics}

Schematics can be seen by going to Panel:Schemats.   You will see a list of
events, together with a listing to the right of each event telling you how
many panels exist in that event by round.  Click on View Schemat to see the
schematics in that event.

You will then see a large listing of school codes (if your league uses
them) together with number codes of competitors.  You will also later see
rooms and judges listed here, once you've assigned them.   You can add new
panels to any round, or delete entire rounds here, by using the Add or
Delete buttons to the top right of any given round.    

If you want to move a competitor for whatever reason, you have two options.
The most common case if you have unbalanced panels because of a drop.  To
fix this, simply click Rebalance for each unbalanced round, and the system
will reassign a student from one of the long panels to the short one, based
on paneling rules.  Or you can click on a student code number and it will
allow you to manually assign that student to another panel, giving you the
screen listing of which panels are best and not allowing you to assign them
to an impossible panel (because of judging).

If you click on a panel letter, this will allow you to manually reassign
judges; remove a judge from the panel by clicking on Remove next to their
name, and then add a new judge by selecting one from the list of clean
judges at the bottom, in the pulldown menu.    If you select Remove and
Fine Judge, the system will automatically add the No-show Judge Fine you
can set on the Setup:Money screen to the school's bill, which you can then
print out a new invoice for. 

The number next to the judges' name indicates first how many students that
judges' school has in that event, and how many other rounds that judge is
judging.   If you are using the Judge Quality System, it will also list the
letter code of the judge's quality.  
	
	
\section{Paneling Rooms}

Then you can panel rooms.  Go to Panel:Rooms.   You can do this two ways,
allow the system to pick whichever rooms are free, or select the rooms
manually.  Usually I pick rooms for Congress and draw events manually,
along with any other events that need to be in particular rooms, and then
allow the system to auto assign the other rooms.   At any rate, click on
Assign Manually to assign rooms by event and by hand.  Click on Automatic
to let the system take care of it for you. 
 	
\section{Printouts}

All of the various printouts you might want to have are under
Panel::Printouts.  You can print out manual tabbing sheets, schematics,
judging masters, and the like.  Having a spare copy of these materials
would not be a bad idea.    

You can print out master ballots that will list out the speaker codes in
order of the contestants in each round, and have information about the
judges listed along the top.   Alternately, you can also print out ballot
labels to affix to your own master ballots or ballot packets.

The ballot labels print on Avery ballot labels of the standard size that
comes in sheets of 3x10.  They will print out judge name and code, the room
and time of the round, the event, and a list of competitors in that round,
in order of speaker if you set speaker order, and in order of code if you
do not.   It will also place asterisks next to competitor codes that are
double entered, if your tournament permits double entry.

Also, be sure to test the ballot labels from this screen to see that the
formatting out of your printer is the same as the ballot labels' layout.
Printing is a right royal pain, and getting the labels to line up correctly
is not always possible.  However, if you are printing your labels via Adobe
products, there will often be a "helpful" option to Fit Document or Shrink
Pages to Fit or something like that.   Be sure this option is {\emph
disabled} or else your label printouts won't even be close to lining up;
without the option, they will line up on most printers.

\section{Checking for Disasters}

After the Printouts screen there is a rather useful Disasters function
listed.   This screen will analyze your tournament for a set of common
problems which you should be aware of, such as events with unbalanced
panels (usually caused by drops), judges who are double booked somehow, and
panels without rooms or judges.  This is a good screen to check before you
release your schematics and information to the tournament.     

\chapter{Tabulation}

Tabulating your tournament happens in a few stages; this is the brunt of
what you'll be doing during your tournament day.  Entering ranks requires a
few people; my rule of thumb in determining the amount of staff required to
run your tournament is usually around 100 ballots per round per tabber if
you are not using  master ballots, and around 150-200 ballots per round per
tabber if you are.  Master ballots tend to make typing much faster.  It is
a good idea to not count your tournament director or someone else as a
typist;  they will be needed on a computer to answer questions and fix
problems during the day. 

\section{Entering Ranks}

To enter ranks, go to the first option under the Tabbing menu, Enter Ranks.
Select the time slot that the round occurs in on the right hand side of the
screen, and enter the judge's numerical judge code.   The competitors that
the judge has judged should appear. 

You will only be prompted to enter quality points if you have specified
quality points as one of your tiebreakers.  Otherwise, enter the ranks of
the students one at a time.  The program will automatically advance the
cursor as you enter each rank.

Once you have finished entering a judge, you can enter the code of the next
judge at the bottom of the screen, and the program will save your current
ballot and move automatically to the next one.

\section{Auditing Ballots}

Ballots are not finished when they are entered.   The system requires you
to audit each ballot as well as entering it.  Give the ballot to someone
else to ensure a minimum of errors.

Auditing is done on the same screen as entering happened in.   Click on
"Audit" in the lower right hand corner and you will be in Audit mode.
Audit mode simply allows you to look at each rank and confirm the data on
the screen matches the data on the ballot.  If it is the same, then simply
type in the number of the next ballot to be entered, and click on Audit.   

This screen will also allow you to make changes; if you need to make a
change, make the change in the screen, and instead of hitting "Audit" hit
"Save Ranks" at the bottom instead.   {\emph If you save ranks using this
method, the ballot will not be marked as audited}.  You will instead have
to have someone else re-audit your changes.  The idea is the system never
allows a rank to go forward without two people confirming the rank is
correct.

\section{Monitoring your tournament}

The Tab:Status menu allows you to monitor the progress of your tournament.
Selecting each round will allow you to track the progress of your data
entry.   Each round will list the outstanding ballots to be ranked, and the
ballots that have been ranked but not audited.  If an event is not showing
itself as being available for advancing to eliminations, check here and you
may find which ballot is holding you up.

\section{Breaks}

Break rounds, or elimination rounds, can be calculated after you're
finished entering ranks for a certain event.  When you go to Tab:Breaks,
any event that is ready to advance to eliminations will appear in a column
to the left.  Check the boxes next to events that you wish to advance to
the next round.

You can dynamically assign rounds to a given time slot at this time, and
determine yourself which kind of elimination you want to advance the system
to next.   You can advance an arbitrary number of students to a arbitrary
number of panels in any given round.   

At this time the system only will advance kids to elimination rounds
paneled based on power protection; the seeding will be followed and no
other attempt to move kids based on school matching will be used.
However, elimination rounds will show the seeding of each student on the
tournament director's schematic screen (NOT the printed schematic) so you
can adjust the round to your contentment after the fact.

If you check the "View rankings" page, the system will display after doing
the elimination rounds the totals and tiebreakers used by the system in
order to advance students.  Students advancing will appear in boldface.
Each student code will also be a link; if you need to advance a student
manually into a round, simply click on their link.  Then, under the listing
of panels the student has competed in will appear a link to Assign Student
Manually in the elimination round you've just created; click this link and
you will be able to manually select an elimination round for the student to
appear in.

If you horribly botch the breaking process somehow and just want to do it
again, go to Panel:Schemat for the round, and click on Delete above the
round you wish to delete.   This will delete the whole round (and give you
a big scary warning about the fact) so be sure you delete the right round
(the elim you want to, not a prelim you've already entered results for, or
something like that.)

\section{Elimination Round Judges}

You can choose judges for elimination rounds several ways.  The first and
easiest is to simply allow the computer to do it for you, by going to
Panel:Judges, and then selecting the group you wish to panel, and then
paneling only the rounds happening in the elimination round you wish to
panel.

\chapter{Results}

Finally your tournament is done, your finals are over, and you're ready to
go the hell home.   Congratulations.

The majority of the results you can view or print out are under
Results:Results, simply enough.   There you can see a variety of results
pages that will tell you how kids are doing by event, by school, or by
sweepstakes standings.   Use this to double check during the day that the
program is proceeding, snoop on your kids' success, and so on.   Just be
sure that you select the proper basis for viewing the order of the kids;
viewing kids in order by final round score is pretty nonsensical before the
final is done; you'll want to see them by prelims or elims at that stage in
the game.

You may notice in many of these screens that some ranks will be followed by
a different rank in parentheses.  This indicates that the rank was adjusted
for some reason or another.   For instance, if you are truncating all ranks
to the size of the smallest panel of that event, and the smallest panel is
5, then any ranks of 6 will appear as 5 (6).  That indicates that the 5 was
counted, but the 6 is the rank which was earned by the student.

You can also use this to print out full results, which will list every
student in order with a full accounting of their rankings.    One useful
printout is Awards, which will print out names, schools and placement of
any student who is receiving an award, to read off of at awards.

\section{Codebreaker}

This screen is useful in ballot sorting; it simply lists out each
competitor code followed by their school code.   Use this to sort through
ballots.

\section{Publish}

The Publish screen will make final results available to the coaches of each
school.  It will not publish full results on your website; once coaches log
into their account, their own students' scores will be accessible from
their registration screen.   If you want to publish full  results it's best
to download the PDF file of the full results and post that on the web
somewhere.

\section{Go Home}

That's it.  It's over.  Go home.  You're done.

\end{document}
